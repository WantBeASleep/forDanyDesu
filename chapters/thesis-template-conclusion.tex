\chapter*{Заключение}
\addcontentsline{toc}{chapter}{Заключение}

В результате проведенного анализа, моделирования и проектирования реализованна система ”Интеллектуальный ассистент врача”


В аналитической части выполнен анализ архитектурных стилей построения систем, с учетом требований к динамической расширяемости отдельных модулей системы, выбрана микросервисная архитектура.
Были проанализированы основные архитектурные паттерны построения приложений. Clean Architecture за счет преобраладания SOLID, разбиения на слои и преимуществ Dependency inversion выбрана как основа написания микросервисов.


В теоретической части была выделенны основные предметные области системы, смоделирована система состоящая из компонентов ответственных за предметные области.
Каждый компонент смоделирован в отдельности с учетом схемы базы данных, а также построена модель взаимодействия компонентов меж собой в рамках системы. 
Модели соответствуют и учитывают все функциональные требования к модулям системы и системе в целом.


В третей главе разработана итоговая архитектура системы, выбраны инструменты и технологии для реализации.


В четвертой главе представлена реализация системы, описаны основные детали реализации, приведены фрагменты кода. Описаны методы и сценарии тестирования системы, разработаны
подходы для воспроизводимости сценарием проведения тестирования.


\textbf{В рамках ВКР} были проведены анализ проблемной области, моделирование системы, проектирование системы, программная реализация и тестирование серверного 
приложения «Интеллектуальный ассистент врача УЗИ».

%%% Local Variables:
%%% TeX-engine: xetex
%%% eval: (setq-local TeX-master (concat "../" (seq-find (-cut string-match ".*-3-pz\.tex$" <>) (directory-files ".."))))
%%% End:
