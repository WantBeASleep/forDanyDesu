\chapter*{Заключение}
\addcontentsline{toc}{chapter}{Заключение}

Данная работа посвящена анализу, моделированию, проектированию и реализации серверного приложения «Интеллектуальный ассистент врача УЗИ» в рамках медицинской информационной системы.


В ходе выполнения работы были решены все поставленные задачи и получены следующие результаты:
\begin{enumerate}
    \item Обоснована необходимость создания интеллектуального ассистента врача, направленного на поддержку клинических решений в области ультразвуковой диагностики. Проведён анализ архитектурных стилей построения программных систем с учётом требований к масштабируемости, модульности и устойчивости. В результате обоснован выбор микросервисной архитектуры как оптимального подхода для обеспечения гибкости и возможности расширения функционала системы.
    \item Проанализированы современные архитектурные паттерны построения приложений. В качестве архитектурной основы была выбрана Clean Architecture, обеспечивающая логическое разделение слоёв, независимость бизнес-логики от деталей реализации и высокую тестируемость. Установлено, что соблюдение принципов SOLID и внедрение инверсии зависимостей способствуют повышению надёжности и сопровождаемости кода.
    \item Определены ключевые предметные области системы. Выполнено детальное моделирование компонентов системы, включающее проектирование структуры базы данных, а также построение модели взаимодействия между модулями. Разработанные модели учитывают все функциональные требования и обеспечивают логическую целостность системы.
    \item Построена итоговая архитектура приложения, определён стек технологий и инструментов, необходимых для реализации. Обоснован выбор языков программирования, библиотек и фреймворков, подходящих для задач, связанных с серверной обработкой и интеграцией в клинические системы.
    \item Выполнена программная реализация серверного приложения, приведены основные технические решения и фрагменты кода, отражающие реализацию ключевых сценариев. Разработаны методы и сценарии тестирования, обеспечивающие воспроизводимость и надёжность полученных результатов. Подтверждена готовность приложения к использованию в составе медицинской информационной системы.
    \item В результате выполнения работы была создана полнофункциональная серверная система, способная служить основой для построения интеллектуального ассистента врача УЗИ, с возможностью дальнейшего расширения и интеграции в реальную клиническую практику.
\end{enumerate}

%%% Local Variables:
%%% TeX-engine: xetex
%%% eval: (setq-local TeX-master (concat "../" (seq-find (-cut string-match ".*-3-pz\.tex$" <>) (directory-files ".."))))
%%% End:
