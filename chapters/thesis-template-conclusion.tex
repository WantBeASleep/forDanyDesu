\chapter*{Заключение}
\addcontentsline{toc}{chapter}{Заключение}

В аналитической части: 



\begin{enumerate}
	\item Выполнен анализ архитектурных стилей построения систем, с учетом требований к динамической расширяемости отдельных модулей системы, выбрана микросервисная архитектура с применением API Gateway.
	\item Проанализированы основные архитектурные паттерны построения приложений. Clean Architecture за счет преобраладания SOLID, разбиения на слои и преимуществ Dependency inversion выбрана как основа написания микросервисов.
	\item Проанализированных основные инструменты для разработки системы, произведены сравнения аналогов\\
        \item Для синхронного взаимодействия между микросервисами выбран gRPC, за счет строгой типизации контрактов и производительности за счет сериализации данных.\\
        \item Для асинхронного взаимодействия между микросервисами выбран подход с брокером сообщений, за счет универсальности и отказоустойчивости. В качестве реализации выбрана RedPanda, Kafka совместимое API с наибольшей производительностью.\\
        \item Для хранения структурированных данных выбрана PostgreSql, за счет развитой надежности и удобства\\
        \item Для хранения неструтурированных данных выбрано NoSQL хранилище S3 Minio.\\
        \item В качестве схемы авторизации выбрана схема с JWT токенами, которая позволяет не хранить активное состоянии на серверной стороне.\\
\end{enumerate}


В теоретической части:


Были спроектированы модели Auth Service, Uzi Service и Gateway Service. Объединением всех этих моделей является
модель самой системы. Модели соответствуют и учитывают все функциональные требования к модулям системы и системе в целом.


В проектировании:


В результате были спроектированы модули системы и правила их взаимодействия с учетом выбранных средств разработки
на этапе анализа.


В итоге:


Таким образом, в рамках НИРа было реализовано 4 микросервиса серверного приложения, а также были проведены некоторые типы тестирования реализации. 
В будущем планируется развернуть систему на нескольких физическим машинах с использованием оркестратора контейнеризированных приложений 
kubernetus и повторно провести нагрузочное и интеграционное тестирование.

%%% Local Variables:
%%% TeX-engine: xetex
%%% eval: (setq-local TeX-master (concat "../" (seq-find (-cut string-match ".*-3-pz\.tex$" <>) (directory-files ".."))))
%%% End:
