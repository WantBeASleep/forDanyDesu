\title{Разработка серверного компонента клиент-серверной программной системы диагностики щитовидной железы по ультразвуковым снимкам}

\taskdate{26.12.2024}

\projecttasks{%
  \projecttask{\bfseries\projecttasknum}{\bfseries Аналитическая часть}{}{}{}%
    % (указываются предмет и цели анализа)
  \projecttask{\projectsubtasknum}%
  {Анализ и сравнение архитектур построения серверных приложений}%
  {}%
  {}{}%
  \projecttask{\projectsubtasknum}%
  {Анализ и сравнение паттернов проектирования модулей системы}%
  {}%
  {}{}%
  \projecttask{\projectsubtasknum}%
  {Анализ инструментов применяемых для реализации системы}%
  {}%
  {}{}%
  \projecttask{\bfseries\projecttasknum}{\bfseries Теоретическая часть}{}{}{}%
    % (указываются используемые и разрабатываемые модели, методы, алгоритмы)
  \projecttask{\projectsubtasknum}%
  {Разработка моделей микросервисов}%
  {}%
  {}{}%
  \projecttask{\projectsubtasknum}%
  {Разработка общей модели системы}%
  {}%
  {}{}%
  \projecttask{\bfseries\projecttasknum}{\bfseries Инженерная часть}{}{}{}%
    % (указывается, что конкретно необходимо спроектировать, а также используемые для этого методы, технологии и инструментальные средства)
  \projecttask{\projectsubtasknum}%
  {Проектирование модулей на основе разработанных моделей}%
  {}%
  {}{}%
  \projecttask{\projectsubtasknum}%
  {Проектирование системы на основе разработанных модели взаимодействия}%
  {}%
  {}{}%
  \projecttask{\bfseries\projecttasknum}{\bfseries Реализация}{}{}{}%
    % (указывается, что конкретно должно быть реализовано и протестировано, а также используемые для этого методы, инструментальные средства, технологии)
  \projecttask{\projectsubtasknum}%
  {Реализация системы}%
  {}%
  {}{}%
  }

\taskliterature{
}

% # Подписи

% Для простановки подписи используются слдеющие команды:
% - простая подпись: \sign[<сдвиг>]{<масштаб>}{<FIO>}
% - подпись с датой: \signat[<сдвиг>]{<масштаб>}{<FIO>}{<дата>}
% где
% - <сдвиг> --- необязательный сдвиг подписи по вертикали для правильного
%   расположения относительно строки
% - <масштаб> --- число, используемое для масштарибования изображения подписи
% - <FIO> --- имя файла с подписью, файл должен быть помещен в
%   img/signatures/FIO.png и иметь прозрачный фон
% - <дата> --- дата, которая будет выведена под подписью

% ## Утверждение задания руководителем и студентом
\authortaskapproval{}%
\supervisortaskapproval{}%

% ## Утверждение РСПЗ руководителем, студентом и консультантом
\authorrspzapproval{}%
\supervisorrspzapproval{}%
\consultantrspzapproval{}%

% ## Оценка руководителя за РСПЗ
\supervisorrspzgrade{10}%

% ## Утверждение ПЗ руководителем, студентом и консультантом
\authorpzapproval{}%
\supervisorpzapproval{}%
\consultantpzapproval{}%

% ## Оценка руководителя за ПЗ
\supervisorpzgrade{12}%

%%% Local Variables:
%%% TeX-engine: xetex
%%% eval: (setq-local TeX-master (concat "../" (seq-find (-cut string-match ".*-1-task\.tex$" <>) (directory-files ".."))))
%%% End:
