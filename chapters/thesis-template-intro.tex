\chapter*{Введение}
\label{sec:afterwords}
\addcontentsline{toc}{chapter}{Введение}

Экспертиза щитовидной железы узла является очень сложной и сложной задачей и требует высококвалифицированных медицинских работников. Специальная система классификации EU-TIRADS используется для определения типа узла, основанного на различных особенностях узла, таких как размер, форма и источник эха. Тем не менее, ручная классификация узлов в соответствии с системой EU-TIRADS скучна и может привести к ошибкам.


Из-за широкого использования методов глубокого обучения в различных областях жизни эти методы также используются для решения проблем диагностических эндокринных заболеваний. Чтобы эффективно использовать эти методы в ежедневной работе работников здравоохранения, проводящих эндокринные исследования, необходимо разработать серверные приложения для интуитивного взаимодействия между медицинским персоналом и технологиями искусственного интеллекта.


Целью работы является разработка серверного приложения для упрощения классификации заболеваний в соответствии с EU-TIRADS. Чтобы достичь этой цели, вам необходимо будет разработать архитектуру для вашего приложения для вашего сервера или финансировать существующие, предоставить приложения на единой цифровой платформе российской федерации «ГосТЕХ» и предоставить приложения для измерения скорости обработки и количества запросов, выполняемых одновременно.


Аналогичные технологии уже были реализованы в виде дополнительного программного обеспечения, интегрированного в ультразвуковую диагностику или прикладное программное обеспечение. Однако на данный момент такая технология не представлена ​​на российском рынке. Следовательно, разработка серверных приложений для упрощения классификации заболеваний в EU-TIRADS является важным шагом в разработке эндокринной диагностики.

%%% Local Variables:
%%% TeX-engine: xetex
%%% eval: (setq-local TeX-master (concat "../" (seq-find (-cut string-match ".*-3-pz\.tex$" <>) (directory-files ".."))))
%%% End:
