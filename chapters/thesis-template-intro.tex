\chapter*{Введение}
\label{sec:afterwords}
\addcontentsline{toc}{chapter}{Введение}

Исследование щитовидной железы на наличие узловых образований является крайне сложной и комплексной задачей, 
требующей высокой квалификации медицинских работников. Для определения природы узлов используется специальная система классификации EU-TIRADS, которая основана на различных характеристиках узлов, таких как размер, форма и эхогенность. Однако, ручная классификация узлов по системе EU-TIRADS может быть трудоемкой и приводить к ошибкам.
В связи с широким использованием методов глубокого обучения в различных сферах жизни, эти методы применяются и для решения задачи диагностики эндокринологических заболеваний. Для того, чтобы эти методы были эффективно использованы в повседневной работе медицинских работников, проводящих эндокринологические исследования, необходимо разработать серверное приложение для интуитивного взаимодействия медицинского персонала с технологиями искусственного интеллекта.
Цель работы заключается в разработке серверного приложения для упрощения классификации заболеваний по EU-TIRADS. Для достижения этой цели необходимо разработать архитектуру серверного приложения или доработать уже имеющуюся, развернуть приложение на единой цифровой платформе РФ «ГосТех», а также провести нагрузочное тестирование с замерами скорости обработки и количества одновременно выполняющихся запросов.
Подобные технологии уже внедряются в качестве дополнительного программного обеспечения, интегрированного в аппараты ультразвуковой диагностики или в качестве прикладного программного обеспечения. Однако, на данный момент на российском рынке подобные технологии не представлены. Поэтому, разработка серверного приложения для упрощения классификации заболеваний по EU-TIRADS является важным шагом в развитии эндокринологической диагностики.
В первом разделе данной работы проводится анализ проблемной области, исследование существующих инструментов для создания серверного приложения, анализ имеющейся архитектуры программной системы, рассматриваются трудности, возникающие при миграции серверного приложения с одного языка программирования на другой. Также ставятся цели и задачи НИР.
Во втором разделе проводится сравнительный анализ инструментов для создания серверного приложения и выбор наиболее подходящего, а также анализируются необходимые изменения в базе данных.
В третьем разделе описана разработка выбранной архитектуры приложения, компоненты и их взаимодействие между друг другом и интеграция с внешними системами.
В четвертом разделе произведена программная реализация серверного приложения и проведено его тестирование.

%%% Local Variables:
%%% TeX-engine: xetex
%%% eval: (setq-local TeX-master (concat "../" (seq-find (-cut string-match ".*-3-pz\.tex$" <>) (directory-files ".."))))
%%% End:
