\pagestyle{empty}
\input{title/blank-header}

\begin{center}
  {\Large{\textbf{Задание на УИР}}}

  \large
 
  Студенту гр. \theauthorgroup{} \theauthorfulldative
\end{center}

\begin{center}
  \uppercase{\textbf{\large{}Тема УИР}}

  {\Large\thetitle}

  \vskip 1em

  \uppercase{\textbf{\large{}Задание}}
\end{center}

{\linespread{1.0}
  \footnotesize
  \noindent
  \begin{longtable}{|p{.5cm}|p{250pt}|>{\centering\arraybackslash}p{2cm}|>{\centering\arraybackslash}p{2cm}|>{\centering\arraybackslash}p{2cm}|} \hline
  \multicolumn{1}{|>{\centering\arraybackslash}p{0.5cm}|}{№\par п/п} & \multicolumn{1}{c|}{Содержание работы} & Форма отчетности & Срок исполнения & Отметка о выполнении\par {\scriptsize Дата, подпись} \\\hline
\projecttask & Аналитическая часть &&& \\\hline
% (указываются предмет и цели анализа)
\projectsubtask & Изучение и сравнительный анализ … с целью… 
  &
  &
  &
  \\\hline
\projectsubtask & Изучение и анализ … для… 
  &
  &
  &
  \\\hline
\projectsubtask & Анализ … применительно к задачам… 
  &
  &
  &
  \\\hline
\projectsubtask & Анализ возможностей… (для…, применительно к…, и т.п.) 
  &
  &
  &
  \\\hline
\projectsubtask & Оформление расширенного содержания пояснительной записки (РСПЗ) & Текст РСПЗ & 20.10.2020 & \\\hline%
\projecttask & Теоретическая часть &&& \\\hline
% (указываются используемые и разрабатываемые модели, методы, алгоритмы)
\projectsubtask & Используется … (модель, метод, алгоритм(ы)…) Модель/ алгоритм/метод... 
  &
  &
  &
  \\\hline
\projectsubtask & Выбор/разработка… 
  &
  &
  &
  \\\hline
\projectsubtask & Разработка… 
  &
  &
  &
  \\\hline
\projectsubtask & Модификация… (алгоритма, модели, и т.п.) для … 
  &
  &
  &
  \\\hline
\projectsubtask & Адаптация … для… 
  &
  &
  &
  \\\hline
\projecttask & Инженерная часть &&& \\\hline
% (указывается, что конкретно необходимо спроектировать, а также используемые для этого методы, технологии и инструментальные средства)
\projectsubtask & Проектирование … (системы, подсистемы, модуля…) 
  &
  &
  &
  \\\hline
\projectsubtask & Использовать методологию проектирования…. 
  &
  &
  &
  \\\hline
\projectsubtask & Разработать архитектуру для… (с учетом требований к…) 
  &
  &
  &
  \\\hline
\projectsubtask & Результаты проектирования оформить с помощью…. При проектировании использовать язык… (например, IDEF, или UML) 
  &
  &
  &
  \\\hline
\projecttask & Технологическая и практическая часть &&& \\\hline
% (указывается, что конкретно должно быть реализовано и протестировано, а также используемые для этого методы, инструментальные средства, технологии)
\projectsubtask & Реализовать… (систему, подсистему, модуль…)
  & Исполняемые файлы, исходный текст 
  &
  & \\\hline%
\projectsubtask & Протестировать… с помощью… 
  &
  &
  &
  \\\hline
\projectsubtask & Разработать тестовые примеры для… & Исполняемые файлы, исходные тексты тестов и тестовых примеров & & \\\hline%
\projectsubtask & Реализация должна иметь форму/обладать качествами... 
  &
  &
  &
  \\\hline
\projectsubtask & Ожидаемым результатом является программная система/программный комплекс/программное обеспечение… со следующими отличительными характеристиками… 
  &
  &
  &
  \\\hline
\projectsubtask & При реализации использовать технологию/платформу… 
  &
  &
  &
  \\\hline
\projecttask & Оформление пояснительной записки (ПЗ) и иллюстративного материала для доклада. & Текст ПЗ, презентация & 15.12.2020 & \\\hline
\end{longtable}
}
\refsection
\nocite{Sychev}
\nocite{Sokolov}
\nocite{Gaidaenko}
\begin{center}
  \uppercase{\textbf{\large{}Литература}}
\end{center}
\printbibliography[heading=none]
\endrefsection

\vfill

{\noindent\linespread{2.0}
  \begin{tabularx}{\linewidth}{p{140pt}XXX}
    Дата выдачи задания: & Руководитель & \hrulefill & \theauthor \\
    01.09.2020           & Студент      & \hrulefill & \thesupervisor \\
  \end{tabularx}
}
